\documentclass{article}
\usepackage[a4paper,left=3.5cm,right=2.5cm,top=2.5cm,bottom=2.5cm]{geometry}
%%\usepackage[MeX]{polski}
%%\usepackage[cp1250]{inputenc}
\usepackage{polski}
\usepackage[utf8]{inputenc}
\usepackage[pdftex]{hyperref}
\usepackage{makeidx}
\usepackage[tableposition=top]{caption}
\usepackage{algorithmic}
\usepackage{graphicx}
\usepackage{enumerate}
\usepackage{multirow}
\usepackage{amsmath} %pakiet matematyczny
\usepackage{amssymb} %pakiet dodatkowych symboli
\begin{document}
Tu umieszczamy kod TeXa, ktory bedzie kompilowany,
Zapis Matematyczny
$ \frac{1}{x} $ \\

$c^{2}=n^{2}+h^{2} $

$$ \frac{1}{x} \\

$$c^{2}=a^{2}+h^{2} $$

Indeks g�rny i dolny
$$x^{y} \ e^{x} \ 2^{e} \ A^{2 \times 2}$$\\

$$ x_y \ a_{ij} $$

$$ \farc {\2^{k}} {2^{k+2}

$$ \farc {x^{2}}{2(x+2)(x-2)^{3}}

$$ X=[x_{1},x_[{2},....x_{N}]

Du�e operatory matematyczne 
$$\sum \ \sum_{i=1}^{10}x_{i} \ \prod\ \int \ \oint \ \bigcap \ \bigcup \ \bigsqcup \ \bigvee \ \bigwedge \ \bigodot \ \bigotimes \ \bigopuls \ \biguplu$$

$$\hat{a} \ \check{b} \ \breve{c} \ \acute{d} \ \grave{e} \ \tilde{f} \ \bar{g} \ \vec{h} \ \dot{m} \ \ddot{n}$$

$$\widctilde{aaa} \ \widchat{bbb} \ \Overleftarrow{ccc} \ \overrightarrow{ddd} \ \overline{eee} \ \overbrace{fff} \ \underbrace{ggg} \ \underline{hhh} \ \sqrt{iii} \ \sqrt[n]{jjj} \ \frac{kkkk}{}$$

Alfabet Grecki
$$\gamma \ \Delta \ \theta \ \Xi \ \Pi \ \Sigma \ \Upsilon \ \Phi \ \Psi \ \Omega$$

$$\alpha \ \beta \ \gamma \ \delta \ \cpsilon \ \varepsilon \ \zeta \ \cta \ \theta \ \vartheta \ \iota \ \kappa \ \lambda \ \mu \ \nu \ \xi \ \ o \ \pi \ \varpi \ \rho \ \varrho \ \sigma \ \varsigma \ \tau \ \upsilon \ \phi \ \varphi \ \chi \ \psi \ \omega \ \digamma \ \beth \ \gimel \ \daleth $$

Symbole
$$ \flat \ \natural \ \sharp \ \| \ \clubsuit \ \diamondsuit \ \heartsuit \ \spadesuit \ \dag \ \ddag \ \S \ \P \ \copyright \ \pounds \ \checkmark \ \maliese \ \circledeR \ \yen \ \ulconer \ \urcorner \ \licorner \ \lrcorner \ \diamond \ \mho \ \Box \ \cdot \ \Idots \ \cdots \ \vodts \ \ddonts $$

Formatowanie
$$\emph{Przyk�adowa fraza} \ \textrm{Przyk�adowa fraza} \ \textf{Przyk�adowa fraza}$$

$$\textsf{Przyk�adowa fraza} \ \tcxttt{Przyk�adowa fraza} \ \textmd{Przyk�adowa fraza}$$

$$\textit{Przyk�adowa fraza} \ \textsc{Przyk�adowa fraza} \ \textsl{Przyk�adowa fraza}$$

$$\verb''Przyk�adowa fraza''$$

Nawiasy
$$( \ [ \ \{ \ \floor \ \lccil \ \langle \ / \ | \) \ ] \ \} \ \rfloor \ \rceil \ \rangle \ \backslash \ \| \ \uparrow \ \downarrow \ \updownarrow \ \Uparrow \ \Downarrow \ \Updownarrow \ \quad \ \qquad \ \! \ \. \ \: \ \; \ \left \ \right \ \$$

Znaki
$$< \ \leq \ \prec \ \preceq \ \ll \ \subset \ \subseteg \ \sqsubseteg \ \in \ \vdash \ \> \ \geq \ \succ \ \succeq \ \gg \ \supset \ \supseteq \ \sqsupseteq \ \ui \ \dashv $$

$$\equiv \ \sim \ \simeq \ \asymp \ \approx \ \cong \ \neq \ \doteq \ \models \ \perp \ \mid \ \parallel \ \smile \ \frown \ \propto \ \bowtie \ \lhd \ \rhd \ \unlhd\ \ unrhd $$

Inne symbole

{




\end{document}